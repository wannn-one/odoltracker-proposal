\begin{center}
  \large\textbf{ABSTRAK}
\end{center}

\addcontentsline{toc}{chapter}{ABSTRAK}

\vspace{2ex}

\begingroup
% Menghilangkan padding
\setlength{\tabcolsep}{0pt}

\noindent
\begin{tabularx}{\textwidth}{l >{\centering}m{2em} X}
  Nama Mahasiswa    & : & \name{}         \\

  Judul Tugas Akhir & : & \tatitle{}      \\

  Pembimbing        & : & 1. \advisor{}   \\
                    &   & 2. \coadvisor{} \\
\end{tabularx} 
\endgroup

% Ubah paragraf berikut dengan abstrak dari tugas akhir
% Bahasa isyarat merupakan bahasa yang direpresentasikan dalam gerakan tangan dan ekspresi wajah. Tunarungu menggunakan bahasa isyarat sebagai bahasa komunikasi utama. Dalam berkomunikasi sehari – hari, tunarungu lebih memilih menggunakan BISINDO karena tidak terikat dengan struktur baku bahasa Indonesia dan disertai ekspresi wajah. Menurut GERKATIN terdapat setidaknya 2,9 juta orang penyandang tunarungu. Jumlah penyandang tunarungu yang cukup besar ini tidak diikuti dengan pengetahuan masyarakat umum mengenai bahasa isyarat. Hal ini berdampak pada sulitnya komunikasi tunarungu dengan masyarakat sekitar sehingga adanya keterbatasan dalam peningkatan kualitas hidup mereka. Sistem penerjemah saat ini masih terbatas dalam menerjemahkan dalam bentuk kata saja dan belum adanya upaya dalam membuat sistem yang bersifat inklusif. Pada tugas akhir ini telah dikembangkan sistem penerjemah BISINDO menggunakan arsitektur LSTM. Sistem telah diimplementasikan pada Intel NUC dengan kemampuan dalam menerjemahkan gerakan isyarat secara \emph{real time}. Pengguna dapat membentuk kalimat - kalimat yang umum digunakan sehari - hari dan mengkonversinya ke media suara dengan bantuan gerakan isyarat kontrol. Berdasarkan pengujian yang telah dilakukan, didapat bahwa sistem dapat beradaptasi dengan adanya perbedaan intensitas cahaya, jarak, serta subjek yang berbeda dengan penulis dengan akurasi tertinggi mencapai 100\%. Sistem juga telah dapat berjalan secara \emph{real-time} dengan performa baik pada Intel \emph{Next Unit Computing} (NUC). Sistem ini dapat menjadi solusi dalam mengatasi hambatan komunikasi antara tunarungu dengan khalayak umum.
Penggunaan truk sebagai moda transportasi barang di Indonesia terus meningkat, namun pelanggaran \emph{ODOL} (\emph{Overdimension Overloading}) pada truk menjadi salah satu penyebab utama kecelakaan lalu lintas. Penelitian ini bertujuan mengembangkan sistem deteksi kendaraan overdimension menggunakan teknologi \emph{deep learning}, khususnya \emph{Convolutional Neural Network} (CNN), yang dapat berjalan secara \emph{real-time} di \emph{edge device}. Sistem ini juga diintegrasikan dengan sistem pemantauan lalu lintas dan dilengkapi fitur notifikasi otomatis kepada pihak berwenang saat pelanggaran terdeteksi. Dengan peningkatan akurasi dan efisiensi, sistem ini diharapkan mampu memberikan solusi efektif untuk mendeteksi kendaraan \emph{ODOL} di berbagai lokasi.

Kata Kunci: \emph{Overdimension}, \emph{ODOL}, \emph{deep learning}, CNN, \emph{edge device}, sistem deteksi

% Ubah kata-kata berikut dengan kata kunci dari tugas akhir
% Kata Kunci: Tunarungu, BISINDO, LSTM, Intel NUC
