\begin{center}
  \large\textbf{ABSTRACT}
\end{center}

\addcontentsline{toc}{chapter}{ABSTRACT}

\vspace{2ex}

\begingroup
% Menghilangkan padding
\setlength{\tabcolsep}{0pt}

\noindent
\begin{tabularx}{\textwidth}{l >{\centering}m{3em} X}
  \emph{Name}     & : & \name{}         \\

  \emph{Title}    & : & \engtatitle{}   \\

  \emph{Advisors} & : & 1. \advisor{}   \\
                  &   & 2. \coadvisor{} \\
\end{tabularx}
\endgroup

% Ubah paragraf berikut dengan abstrak dari tugas akhir dalam Bahasa Inggris
% \emph{
%   Sign language is represented through hand movements and facial expressions. The deaf community uses sign language as their primary means of communication. In daily interactions, the deaf often prefer using BISINDO, which is not constrained by the strict structure of the Indonesian language and includes facial expressions. According to GERKATIN, there are at least 2.9 million deaf individuals. However, the substantial number of deaf individuals is not matched by public knowledge of sign language. This results in communication difficulties between the deaf and the surrounding community, thus limiting their quality of life improvement. Current translation systems are limited to translating into words only and there has been no effort to create a truly inclusive system. In this thesis, a BISINDO translation system using LSTM architecture has been developed. The system has been implemented on an Intel NUC, capable of translating sign language movements in real time. Users can form commonly used daily sentences and convert them into spoken media with the help of control sign movements. Testing has shown that the system can adapt to different light intensities, distances, and subjects different from the author, with the highest accuracy reaching 100\%. System has shown an excellent realtime performance in Intel Next Unit Computing (NUC). This system can be a solution to overcome communication barriers between the deaf and the general public.}
\emph{The use of trucks as a mode of transportation for goods in Indonesia continues to increase, but overdimension violations (ODOL) on trucks have become a major cause of traffic accidents. This study aims to develop an overdimension vehicle detection system using deep learning technology, specifically Convolutional Neural Network (CNN), that can operate in real-time on edge devices. The system is also integrated with traffic monitoring systems and equipped with an automatic notification feature to alert authorities when violations are detected. By improving accuracy and efficiency, this system is expected to provide an effective solution for detecting ODOL vehicles across various locations}.

\emph{Keywords}: \emph{Overdimension, ODOL, deep learning, CNN, edge device, detection system}

% Ubah kata-kata berikut dengan kata kunci dari tugas akhir dalam Bahasa Inggris
% \emph{Keywords}: \emph{Deaf}, BISINDO, LSTM, Intel NUC.
