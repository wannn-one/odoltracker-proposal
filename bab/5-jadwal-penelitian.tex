\chapter{JADWAL PENELITIAN}
\label{chap:jadwalpenelitian}

% Ubah tabel berikut sesuai dengan isi dari rencana kerja
\newcommand{\w}{}
\newcommand{\G}{\cellcolor{gray}}
\begin{table}[H]
  \captionof{table}{Tabel \emph{timeline} penelitian}
  \label{tbl:timeline}
  \begin{tabular}{|p{3.5cm}|c|c|c|c|c|c|c|c|c|c|c|c|c|c|c|c|}

    \hline
    \multirow{2}{*}{Kegiatan} & \multicolumn{16}{|c|}{Minggu}                                                                       \\
    \cline{2-17}              &
    1                         & 2                             & 3  & 4  & 5  & 6  & 7  & 8  & 9  & 10 & 11 & 12 & 13 & 14 & 15 & 16 \\
    \hline

    % Gunakan \G untuk mengisi sel dan \w untuk mengosongkan sel
    Pengambilan data          &
    \G                        & \G                            & \G & \w & \w & \w & \w & \w & \w & \w & \w & \w & \w & \w & \w & \w \\
    \hline

    Praproses \& pengolahan data           &
    \w                        & \w                            & \G & \G & \G & \G & \G & \G & \w & \w & \w & \w & \w & \w & \w & \w \\
    \hline

    \emph{Testing} \& Analisa data              &
    \w                        & \w                            & \w & \w & \w & \w & \G & \G & \G & \G & \G & \w & \w & \w & \w & \w \\
    \hline

    Evaluasi penelitian \& dokumentasi       &
    \w                        & \w                            & \w & \w & \w & \w & \w & \w & \w & \w & \G & \G & \G & \G & \G & \G \\
    \hline
  \end{tabular}
\end{table}

Pada \emph{timeline} yang tertera di Tabel \ref{tbl:timeline}, terdapat empat kegiatan utama yang akan dilakukan selama 16 minggu. Kegiatan-kegiatan tersebut adalah sebagai berikut:

\begin{enumerate}
  \item \textbf{Pengambilan data} (Minggu 1-3) \\
        Pada tahap ini, dilakukan pengambilan data yang akan digunakan dalam penelitian ini. Data yang diambil adalah data kendaraan \emph{overdimension} yang melintas di gerbang tol.

  \item \textbf{Praproses \& pengolahan data} (Minggu 3-8) \\
        Pada tahap ini, dilakukan praproses dan pengolahan data yang telah diambil. Data yang telah diambil akan diproses dan diolah agar dapat digunakan dalam pelatihan model.

  \item \textbf{\emph{Testing} \& Analisa data} (Minggu 7-11) \\
        Pada tahap ini, dilakukan pengujian model yang telah dilatih. Model yang telah dilatih akan diuji dengan data yang belum pernah dilihat sebelumnya. Selain itu, dilakukan analisa terhadap hasil pengujian.

  \item \textbf{Evaluasi penelitian \& dokumentasi} (Minggu 11-16) \\
        Pada tahap ini, dilakukan evaluasi terhadap penelitian yang telah dilakukan. Selain itu, dilakukan dokumentasi terhadap penelitian yang telah dilakukan.
\end{enumerate}