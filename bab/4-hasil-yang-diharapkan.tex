\chapter{HASIL YANG DIHARAPKAN}
\label{chap:hasilyangdiharapkan}

% Ubah bagian-bagian berikut dengan hasil yang diharapkan dari penelitian

\section{Hasil yang Diharapkan dari Penelitian}

Berdasarkan hasil pengujian yang telah dilakukan dalam tugas akhir ini, diharapkan penelitian ini menghasilkan suatu sistem deteksi kendaraan \emph{overdimension} dengan model yang di-\emph{deploy} di edge device dengan akurasi dan kecepatan yang tinggi. Sistem ini diharapkan dapat digunakan pada gerbang tol untuk mendeteksi kendaraan \emph{overdimension} yang melintas.

% \begin{enumerate}
%   \item Membuat model deteksi kendaraan \emph{overdimension} yang dapat digunakan pada gerbang tol.
%   \item Membuat sistem deteksi kendaraan \emph{overdimension} yang dapat digunakan pada gerbang tol.
%   \item Membuat sistem deteksi kendaraan \emph{overdimension} yang dapat digunakan pada gerbang tol dengan akurasi yang tinggi.
%   \item Membuat sistem deteksi kendaraan \emph{overdimension} yang dapat digunakan pada gerbang tol dengan kecepatan deteksi yang tinggi.
% \end{enumerate}

\section{Hasil Pendahuluan}

Sampai saat ini, telah dilakukan studi literatur mengenai deteksi kendaraan \emph{overdimension} dan \emph{object detection}. Selain itu, telah dilakukan pengumpulan data dan pengolahan data untuk melatih model deteksi kendaraan \emph{overdimension}.