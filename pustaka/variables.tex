% Atur variabel berikut sesuai namanya

% nama
\newcommand{\name}{Ikhwanul Abiyu Dhiyya'ul Haq}
\newcommand{\authorname}{Dhiyya'ul Haq, Ikhwanul Abiyu}
\newcommand{\nickname}{Ikhwan}
\newcommand{\advisor}{Dr. Arief Kurniawan, S.T., M.T.}
\newcommand{\coadvisor}{Dr. Diah Puspito Wulandari, S.T., M.Sc.}
\newcommand{\examinerone}{Dr. Eko Mulyanto Yuniarno,S.T., M.T.}
\newcommand{\examinertwo}{Arta Kusuma Hernanda, S.T., M.T.}
\newcommand{\examinerthree}{-}
\newcommand{\headofdepartment}{Dr. Arief Kurniawan, S.T., M.T.}

% identitas
\newcommand{\nrp}{5024 211 048}
\newcommand{\advisornip}{19740907 200212 1 001}
\newcommand{\coadvisornip}{19801219 200501 2 001}

\newcommand{\examineronenip}{19680601 199512 1 009}
\newcommand{\examinertwonip}{19962023 11024}
\newcommand{\examinerthreenip}{-}
\newcommand{\headofdepartmentnip}{19740907 200212 1 001}

% judul
\newcommand{\tatitle}{{SISTEM DETEKSI KENDARAAN OVERDIMENSI SECARA \emph{REAL-TIME} DI GERBANG TOL MENGGUNAKAN SSD MOBILENETv2 BERBASIS \emph{EDGE DEVICE}}}
\newcommand{\engtatitle}{\emph{REAL-TIME OVERDIMENSION VEHICLE DETECTION SYSTEM AT TOLL GATES UTILIZING SSD MOBILENETv2 ON EDGE DEVICES}}

% tempat
\newcommand{\place}{Surabaya}

% jurusan
\newcommand{\studyprogram}{Teknik Komputer}
\newcommand{\engstudyprogram}{Computer Engineering}

% fakultas
\newcommand{\faculty}{Teknologi Elektro dan Informatika Cerdas}
\newcommand{\engfaculty}{Intelligent Electrical And Informatics Technology}

% singkatan fakultas
\newcommand{\facultyshort}{FTEIC}
\newcommand{\engfacultyshort}{FIEI}

% departemen
\newcommand{\department}{Teknik Komputer}
\newcommand{\engdepartment}{Computer Engineering}

% kode mata kuliah
\newcommand{\coursecode}{EC234701}
